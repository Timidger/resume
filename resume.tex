\documentclass[overlapped,line]{res}
\topmargin=-.74in
\oddsidemargin -.5in
\evensidemargin -.5in
\textheight=11in
\textwidth=6.62in
\resumewidth=7.2in
\hoffset=.5in
\sectionwidth=0.5in
\newsectionwidth{0in}
\sectionskip=0.1in

\renewcommand{\sectionfont}{\bf}


\begin{document}

\moveleft.5\hoffset\centerline{\huge\bf Preston Carpenter}
\vspace{.2em}

\moveleft.5\hoffset\centerline{apragmaticplace@gmail.com $|$ 417-438-7110}
\moveleft.5\hoffset\centerline{linkedin.com/in/timidger $|$ github.com/Timidger $|$ timidger.github.io }
\vspace{1em}

\begin{resume}

\section{\underline{EXPERIENCE}}

\textbf{Amazon} $|$ Seattle, WA\hfill April 2020 - Present\\
{\sl Software Engineer II}
\begin{itemize} \itemsep -2pt
	\item Successfully led the migration of the global Amazon Linux AMI 
		release process to Step Functions using CDK
	\item Automated tasks in Python and Rust %to run on a distributed
		%network to release external and internal operating system images
		%in all AWS regions and partitions
	\item Centralized the confusing AMI release workflow into a
		Python CLI so operations and data can be managed in a single
		location
	%\item Automating AMI releases with Python, Bash, and Rust for Amazon Linux Machinery team
	\item Packaged critical software components and was release manager for sensitive embargo'd operating system updates
\end{itemize}

\textbf{Starry} $|$ Boston, MA\hfill February 2019 - April 2020\\
{\sl System Software Engineer}
\begin{itemize} \itemsep -2pt
    \item Wrote Rust and C++ firmware for embedded devices that power Starry internet
    \item Implemented backbone fiber network fail-over with Python and networkd
    \item Designed and produced SSH-based remote device support system for technicians, using Rust and gRPC
\end{itemize}

\textbf{Google} $|$ Montreal, QC\hfill August 2018 - November 2018\\
{\sl Software Developer Intern}
\begin{itemize} \itemsep -2pt
	\item Implemented removing malware extensions in the Chrome Cleanup
		Tool%, a program that secures millions of Chrome installs on Windows computers silently every week
    \item Collaborated with other engineers at a Chrome security conference and
      offered my knowledge of Rust to discussions
    \item Expanded reporting metrics and updated privacy policy to facilitate unwanted extension fingerprinting
\end{itemize}

\textbf{Microsoft} $|$ Redmond, WA\hfill May 2018 \--- July 2018\\
{\sl Software Engineer Intern}
\begin{itemize} \itemsep -2pt
	\item Developed a SKU document analyzer and reporting tool for Azure's hardware engineers in C\# and HTML
	\item Collaborated with other teams to standardize external OEM and internal datacenter documents for automation
	\item Streamlined the configuration program for a boot loader used in datacenters to be more user friendly
\end{itemize}

\textbf{Intuit} $|$ San Diego, CA\hfill July 2017 \--- December 2017\\
{\sl Software Engineer Co-op}
\begin{itemize} \itemsep -2pt
	\item Java backend developer on ``machine learning as a service'' product that powers Mint and Turbo data insights
	\item Eliminated manual deployment/testing by setting up automated
      continuous integration and deployment in AWS
	\item Won the internal ``Codechella'' hackathon by extending the intranet website to include skills and project search
\end{itemize}

\textbf{Kinto Care} $|$ Boston, MA\hfill July 2016 \--- January 2017\\
{\sl Full Stack Engineer Co-op}
\begin{itemize} \itemsep -2pt
	\item Developed flagship cross-platform phone app using Javascript with React and Cordova
	\item Integrated and expanded back-end API powered by Flask and Postgres
	%\item Contributed to key design decisions through service research and
    %  weekly Agile sprint meetings
\end{itemize}

\textbf{Beechwood Software} $|$ Boston, MA \hfill January 2017 \--- July 2017\\
{\sl Software Engineer Intern} \hfill November 2015 \--- July 2016
\begin{itemize}  \itemsep -2pt
	\item Contributed C++ code to the open-source IoT (Internet of Things) Alljoyn framework in collaboration with the Allseen Alliance and Linux Foundation
	%\item Devised smoothing algorithms to calculate distance between a wide range of different quality bluetooth devices
	\item Assembled and tested the prototype presented at CES in Las Vegas using Rasberry Pis and Nest thermostats
\end{itemize}
\noindent\makebox[7.15in]{\rule{7.15in}{0.4pt}}


\section{\underline{PROJECTS}}
\textbf{Way Cooler (Wayland Window Manager)}
\begin{itemize}  \itemsep -2pt
	\item Tiling wayland window manager with 6,000+ downloads and 2,000+ stars on Github
	\item Implemented a programmable interface using embedded Lua to extend
      a basic compositor core
    \item Originally written in Rust, later rewritten in C
\end{itemize}

\section {\underline{COMPUTER KNOWLEDGE}}
{\textbf{Languages}: \hskip 1.4em Python, Rust, C, Go, Javascript, Typescript, C++, Bash, Lua \\
\textbf{Technologies}: \hskip 0.3em AWS, CDK, gRPC, Protobuf, NGINX, systemd, Git, 
Wayland, Docker, Flask, Postgres
\vspace{-.7em}
}

% load-bearing
\noindent\makebox[7.15in]{\rule{7.15in}{0.4pt}}

\section{\underline{EDUCATION}} 
 \textbf{Northeastern University} $|$ Boston, MA \hfill September 2014 \--- May 2019 \\
{\sl Bachelor of Science in Computer Science} \\
Relevant Courses: \hskip 1em Compilers, Computer Systems, Programming Languages, \\
\strut\hskip 9.25em Networks and Distributed Systems, Algorithms and Data, Systems Security \vspace{2mm} \\


%\textbf{wlc-rs, wlroots-rs (Safe Rust bindings to C Wayland libraries)} \hfill October 2017 \--- April 2019
%\begin{itemize} \itemsep -2pt
%  \item Designed and implemented a safe interface to wlc and wlroots (its successor)
%  \item Programmed basic proof-of-concept compositors in 100\% safe Rust,
%    including old versions of Way Cooler.
%    \item Thoroughly explored the design space for designing an ergonomic \&
%      safe API around a complicated C library.
%\end{itemize}

%\textbf{AutoWikiaBot (Reddit bot)} \hfill
%\begin{itemize}  \itemsep -2pt
%       \item Developed a Reddit bot in Python to post summaries of Wikia articles that other users linked
%       \item Configured and managed a server running Linux/Debian hosting the bot remotely using SSH
%       \item Adapted a popular Python Wikipedia library to work for Wikia links
%\end{itemize}

\end{resume}
\end{document}
